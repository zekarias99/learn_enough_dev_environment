%= <span class="free"></span>

\noindent ለማንኛውም ፍላጎት ላለው ገንቢ (ወይም በአጠቃላይ ቴክኒካዊ ለሆነ ሰው) በጣም አስፈላጊ ከሆኑት ተግባራት መካከል አንዱ ኮምፒዩተራቸውን ልክ እንደ አንድ \emph{የማበልጸጊያ አካባቢ} በማቀናበር የድር ጣቢያዎችን፣ የድር አፕልኬሽኖችን እና ሌሎች ሶፍትዌሮችን ለማበልጸግ ተስማሚ እንዲሆን ማድረግ ነው፡፡ \ledev\ ፣ ሁሉንም አስፈላጊ ቁሳቁሶች በአንድ ቦታ ላይ በማስቀመጥ \href{http://www.learnenough.com/tutorials}{ዋናውን የበቂ ይማሩ ተከታታይ የስልጠና ትምህርትን} እና የ\rort\/ን ለማሟላት የተቀየሰ ነው፡፡

ይህ የስልጠና ትምህርት የተለያየ ደረጃ ልምድ ላላቸው እና የተካኑ አንባቢያን ላይ ኢላማ ያደረገ፣ አንድ የማበልጸጊያ አከባቢን ለማዘጋጀት የተላያዩ አማራጮችን ይሸፍናል፡፡ በክፍል~\ref{sec:cloud_ide} ውስጥ የሚታየውን አማራጪ ለመጠቀም ከወሰናችሁ፣ አጠቃላይ የሆነ የኮምፒዩተር ዕውቀት ከመኖር በስተቀር፣ ይህንን የስልጠና ትምህርት ለማጠናቀቅ የሚያስፈልግ ምንም አይነት ቅድመ ዝግጅት አያስፈልጋችሁም፡፡ በክፍል~\ref{sec:native_os_setup} ውስጥ ያለውን በጣም ፈታኝ የሆነውን ማዋቀር ተግባራዊ ለማድረግ ከፈለጋችሁ (ይህንንም አብዛኛዎቹ አንባቢያን አንድ ወቅት ላይ እንዲሞክሩት ምክሬን እለግሳለሁ)፣ (\lecl ላይ እንደተሸፈነው) ከዩኒክስ የትእዛዝ መስመር ጋር አንድ መሰረታዊ የሆነ ትውውቅ ሊኖራችሁ ይገባል፤ እንዲሁም (\lete ላይ እንደተሸፈነው) ከአንድ የጽሑፍ አርታኢ እና አንድ ስርዓትን ከማዋቀር ጋር ትውውቅ ሊኖራችሁ እንደሚገባ ይመከራል፡፡

የዚህ የስልጠና ትምህርት የኢ-መጽሐፍ ስሪቶች \href{https://www.softcover.io/email-capture/28fdb94f/learn_enough_dev_environment}{በበቂ ይማሩ} ላይ በነጻ ሲገኙ፣ \href{https://www.amazon.com/Learn-Enough-Dev-Environment-Dangerous-ebook/dp/B01MTEQJ6E}{በአማዞን ላይ ደግሞ በ99 ሳንቲም} ይገኛሉ።

\section{የማበልጸጊያ አከባቢ አማራጮች} % (fold)
\label{sec:dev_environment_options}

በዚህ የስልጠና ትምህርት ውስጥ የምናተኩረው፣ የሚከተሉትን አራት መሰረታዊ የሶፍትዌር ማበልጸጊያ መሳሪያዎችን በመጫን ወይም በማንቃት ላይ ይሆናል (ምስል~\ref{fig:dev_environment}):-

\begin{enumerate}
  \item የትእዛዝ መስመር መናኸሪያ (``ቀፎ'')
  \item የጽሑፍ አርታዒ
  \item የስሪት መቆጣጠሪያ (ጊት)
  \item የፕሮግራም ቋንቋዎች (ሩቢ፣ ወዘተረፈ)
\end{enumerate}

እነዚህ የተለያዩ የሶፍትዌር አይነቶች ላይ የበለጠ መረጃን ለማግኘት ከፈለጋችሁ \lecl\/፣ \lete\/፣ \leg እና \ler ላይ ተመልከቱ።

\begin{figure}
\begin{center}
\image{images/figures/dev_environment.png}
\end{center}
\caption{የአንድ የማበልጸጊያ አካባቢ የተለመዱ ገጸባህርያት።\label{fig:dev_environment}}
\end{figure}

አንድ የማበልጸጊያ አካባቢን በማቀናበር ጊዜ፣ ሁለት የምንመክራቸው ዋና አማራጪ መንገዶች አሉ፤ እነሱም አስቸጋሪነታቸው እየጨመረ በሚሄድ ቅደም ተከተል መሰረት ተዘርዝሯል:-

\begin{enumerate}
  \item የደመና \href{https://en.wikipedia.org/wiki/Integrated_development_environment}{ቅ.ማ.አ (IDE)}
  \item የቤተኛ ስርዓተ ክወና ማዋቀር (Native OS) (ማክ ስ.ክ፣ ሊኒክስ፣ ዊንዶውስ)
\end{enumerate}

ስለዚህ ሁኔታ ልምድ የሌላችሁ ከሆነ፣ የደመና ቅ.ማ.አው ከሌሎቹ የቀለለ የማዋቀር ሂደት ስላለው፣ በሱ እንድትጀምሩ እንመክራለን (ክፍል~\ref{sec:cloud_ide})።

\subsubsection{ቤተኛ ስርዓት} % (fold)
\label{sec:native_system}

አንድ አፕልኬሽንን ለማበልጸግ ሲጀምሩ የደመና ቅ.ማ.አ አማራጩን መጠቀሙ በጣም ጥሩ ቢሆንም፣
መጨረሻ ላይ ግን በቤተኛ ስርዓተ ክወናችሁ ላይ ሶፍትዌሮችን ማበልጸግ መቻሉ በጣም አስፈላጊ ነው።
አለመታደል ሆኖ፣ አንድ ሙሉ በሙሉ ተግባራዊ የሆነ ቤተኛ የማበልጸጊያ አካባቢን ማዋቀሩ አንድ ፈታኝ እና ተስፋ የሚያስቆርጥ ሂደት ሊሆን ይችላል\footnote{ለዚህም ነው ልክ እንደ \rortb ለመሳሰሉ መጽሐፍት መጀመሪያ ላይ አንድ የቤተኛ ማበልጸጊያ ዝግጅትን ማካተቱ መጥፎ ሀሳብ የሆነው፡፡ መጀመሪያ በዋናው ቁሳቁስ/ይዞታ መጀመር እና በኋላ ላይ ደግሞ የቤተኛ ማበልጸጊያ መዋቅሩ ጋር መፋለሙ የተሻለ ነው። የሶስተኛው የ\rortb\ እትምን በአንድ የደመና ቅ.ማ.አ ላይ ለማበልጸግ የተወሰነበት ምክንያትም ይህንን ግንዛቤ እግምት ውስጥ በማስገባት ነበር።} (ቴክኒካዊ ብልሃታችሁን የመጠቀም ሰፊ ዕድልን ዘንግታችሁ ይሆናል (ሳጥን~\ref{aside:technical_sophistication}))፤ እሱ ግን ለማንኛውም ፍላጎት ላለው የቴክኒክ ባለሙያ አንድ አስፈላጊ የመተላለፊያ ስነ-ስርዓት ነው፡፡

ይህንን አስቸጋሪ ፈተና ለመቋቋም፣ በክፍል~\ref{sec:native_os_setup} ውስጥ ለማክክ.ስ (macOS)፣ ለሊኒክስ እና ለዊንዶውስ የቤተኛ ስርዓተ ክወናን እንዴት አድርገን እንደምናቀናብር እንመለከታለን፡፡

\begin{aside}
\label{aside:technical_sophistication}
\heading{ቴክኒካዊ ብልሃት}

\emph{ቴክኒካዊ ብልሃት} ማለት ቴክኒካዊ የሆኑ ችግሮችን በግል የመፍታት ችሎታ ማለት ነው። አንድ የማበልጸጊያ አካባቢን በመጫን አውድ ውስጥ፣ ይህ ማለት አንድ ነገር ከተበላሸ የስህተት መልዕክቱን መጎገሉን ማወቅ፣ አፕልኬሽኑን ማቆም እና አንድ አፕልኬሽንን (ማለት እንደ የመስመር ትእዛዝ ቀፎ ያሉትን አፕልኬሽኖችን) እንደገና ማስጀመር እና እንደዛ ማድረጉ ነገሮችን ያስተካከለ መሆኑን ማየትን ያካትታል።

ብዙውን ጊዜ ለችግሩ አጠቃላይ የሆነ አንድ መፍትሔ ስለማይገኝ፣ አንድ የማበልጸጊያ አካባቢን በማቀናበር ወቅት ብዙ ነገሮች ወዳልተጠበቀ ስህተት ሊያመሩ ይችላሉ። (ሁሉም ነገር እስኪሰራ ድረስ ቴክኒካዊ ብልሃታችሁን መጠቀሙን መቀጠል ይኖብባችኋል፡፡ ችግሩ አልወጣም ካላችሁ ግን ስለሱ ብዙ ልትጨነቁ አይገባም።) ይህ ነገር በሕይወት ዘመናችን ውስጥ በአንድ ወቅት በሁላችንም ላይ ተከስቷል ወይም ሊከሰት ይችላል።

\end{aside}

% subsubsection native_system (end)

% section dev_environment_options (end)

\section{የደመና ቅ.ማ.አ}
\label{sec:cloud_ide}

ካሉት የማበልጸጊያ አካባቢ አማራጮች ሁሉ በጣም ቀላሉ የማበልጸጊያ አካባቢ \emph{የደመና ቅ.ማ.አው (cloud IDE)} ነው፣ ይህም እናንተ የመረጣችሁትን የድር አሳሽ በመጠቀም የምትደርሱበት አንድ \href{https://en.wikipedia.org/wiki/Cloud_computing}{በደመና} ውስጥ የተቀናጀ የማበልጸጊያ አካባቢ ነው። ለማግበር ቀላል ቢሆንም፣ ከዚያ የሚገኘው ስርዓት ግን አንድ የቁም ነገር ደረጃ ያለው የማበልጸጊያ ኮምፕዩተር ነው እንጅ አንድ ተራ መጫወቻ አይደለም። በተጨማሪም፣ እሱን ለመጠቀም የሚያስፈልጋችሁ አንድ ተራ የድር አሳሽ ስለሆነ (ማለት ይህንን ማንኛውም ስርዓተ ክወና ስለሚያቀርበው)፣ የደመና ቅ.ማ.አው በሁሉም ስርዓተ-መሳርያ ላይ በራስሰር ይሰራል።

አንድ የደመና ቅ.ማ.አን ለማስኬድ የሚያገለግሉ በርካታ የንግድ/የሚከፈልባቸው አማራጮች አሉ፣ የ \rort\/ን ለመገንባት የአማዞን የድር አገልግሎቶች አካል ከሆነው \href{http://c9.io/}{ከክላው9}  ጋር ተሻርከናል። ከዚያ የሚገኘው ውጤትም አንድ ለሩቢ ኦንሬይልስ የድር ማበልጸጊያ የሚስማማ አካባቢ ነው፣ በእርግጥ በክፍል~\ref{sec:dev_environment_options} ውስጥ የተጠቀሱትን ሁሉንም ነገሮች ያካትታል፡፡ በተለይም በምስል~\ref{fig:ide_anatomy} ላይ እንደሚታየው፣ የአ.ድ.አ ክላውድ9`ኑ (AWS Cloud9) አንድ የትእዛዝ መስመር መናኸሪያን እና አንድ የጽሑፍ አርታኢን (አንድ የፋይል ስርዓት ዳሳሽን አካቶ) ሰንቆ ይመጣል፡፡ እያንዳንዱ የክላውድ9 የመስሪያ ቦታ አንድ ሙሉ በሙሉ የሚሰራ የሊኒክስ ስርዓትን ስለሚሰጥ፣ የጊት የስሪት ቁጥጥር ስርዓትን፣ እንዲሁም የሩቢን እና የሌሎች በርካታ የፕሮግራም ቋንቋዎችን በራስሰር ያካትታል፡፡

\begin{figure}
\begin{center}
\image{images/figures/ide_anatomy_aws.png}
\end{center}
\caption{የደመና ቅ.ማ.አው ሁለንተናዊ አቀማመጥ።\label{fig:ide_anatomy}}
\end{figure}

የደመና የማበልጸጊያ አካባቢን፣ ጥቅም ላይ ለማዋል፣ የሚያስፈልጉት ሂደቶች የሚከተሉት ናቸው:-\footnote{እንደ አ.ድ.አ ያሉ ጣቢያዎች፣ ባላቸው ይሄ ነው የማይባል እድገት ምክንያት፣ በየጊዜው የምታዩዋቸው መረጃወች ሊለያዩ ይችላሉ፣ ማንኛውንም ልዩነት/አለመጣጣም ለማስተካከል ቴክኒካዊ ችሎታችሁን (ሳጥን~\ref{aside:technical_sophistication}) ተጠቀሙ።}
\begin{enumerate}
\item ክላውድ9 የአማዞን የድር አገልግሎቶች አካል ስለሆነ፣ የአማዞን የድር አገልግሎት \emph{መለያ} ካላችሁ በራስሰር ወደዛው \href{https://aws.amazon.com/}{መግባት}\footnote{https://aws.amazon.com/} ትችላላችሁ፡፡ አንድ አዲስ የክላውድ9 የመስሪያ አካባቢ ለመፍጠር፣ \href{https://console.aws.amazon.com/}{አ.ድ.አ ሰሌዳ} ውስጥ በመግባት፣ በመፈለጊያ ሳጥኑ ውስጥ፣ ``Cloud9'' ብላችሁ ጻፉ፡፡
\item ቀድሞውኑ አንድ የአ.ድ.አ መለያ ከሌላችሁ፣ \href{https://www.railstutorial.org/cloud9-signup}{አንድ የነጻ መለያ ለማግኘት፣ በአ.ድ.አ ክላውድ9 ላይ መመዝገብ} ይኖርባችኋል፡፡ አግባብ የሌለው አጠቃቀምን ለመከላከል፣ አ.ድ.አ በምዝገባ ወቅት አንድ ብቃት ያለው የዱቤ ካርድን ይጠይቃል፤ ነገር ግን የመስሪያ ቦታው፣ (ይህ ጽሑፍ በተጻፈበት ወቅት ለአንድ ዓመት) 100\% ነጻ ነው፤ ስለሆነም ከካርዳችሁ የሚወሰድ ገንዘብ አይኖርም። መለያው እስኪነቃ/እስኪጀምር ድረስ፣ እስከ 24 ሰዓት መጠበቅ ሊኖርባችሁ ይችላል። በእኔ በኩል ግን በአስር ደቂቃ ውስጥ ሁሉ ነገር ዝግጁ ነበረ።
\item አንዴ ክላውድ9 አስተዳደር ገጽ (ምስል~~\ref{fig:cloud9_page_aws}) ላይ ከገባችሁ በኋላ፣ ልክ ምስል~\ref{fig:cloud9_new_workspace} `ን የሚመስል ገጽን እስክታገኙ ድረስ ``Create environment'' የሚለው ላይ ጠቅ ማድረጋችሁን ቀጥሉ። ምስል~\ref{fig:cloud9_new_workspace} ላይ የሚታየውን መረጃ በቅጹ ላይ ሙሉ፡፡ በሚቀጥለው ገጽ ላይ ከሚገኙት ምርጫወች ውስጥ፣ የ Amazon Linux `ን \emph{ሳይሆን} የ \textbf{Ubuntu Server `ን} (ምስል~\ref{fig:ubuntu_server}) ምረጡ። ከዚያ ``Next step'' የሚለውን ጠቅ አድርጉ፡፡ አ.ድ.አ ቅ.ል.አውን እስከሚያቀርብ ድረስ፣ ነባሪ ቅንብሮችን በእሽታ ለማስተናገድ፣ የማረጋገጫ አዝራሩን ጠቅ አድርጉ (ምስል~\ref{fig:cloud9_ide_aws})፡፡ አንድ የሩት\footnote{ሩት በዩኒክስ እና በመሳሰሉት ስርዓቶች ላይ፣ በሁሉም ትእዛዞች እና ፋይሎች ላይ፣ በተፈጥሮው ስልጣን ያለው የተጠቃሚ ስም ወይም መለያ ነው።} ``root'' ተጠቃሚ በመሆናችሁ ምክንያት፣ አንድ የማስጠንቀቂያ መልእክት ልታገኙ ትችላላችሁ፤ ይህንንም ላሁኑ ችላ ብላችሁ ማለፍ ትችላላችሁ፡፡ (በዚህ ጊዜ የማንነት እና የማስተዳደር ባለስልጣን ተጠቃሚ (IAM) ተብሎ የሚጠራውን ተመራጪ ዘዴ ለመተግበር ፍላጎቱ ካላችሁ እንደዛ ማድረግ ትችላላችሁ። ስለዚህ አተገባበር የበለጠ መረጃ ለማግኘት \href{https://www.railstutorial.org/}{\emph{በሩቢ የስልጠና ትምህርት ላይ}} \href{https://www.railstutorial.org/book/user_microposts#sec-image_upload_in_production}{ምዕራፍ 13ን} ተመልከቱ።)
\item በመጨረሻም፣ አንድ የዘመነ የጊት ስሪትን እያስኬዳችሁ መሆኑን አረጋግጡ (ዝርዝር~\ref{code:upgrade_git})።
\end{enumerate}

\begin{figure}
\begin{center}
\image{images/figures/cloud9_page_aws.png}
\end{center}
\caption{የክላውድ9 የአስተዳደር ገጽ።\label{fig:cloud9_page_aws}}
\end{figure}

\begin{figure}
\begin{center}
\image{images/figures/cloud9_name_environment.png}
\end{center}
\caption{በአ.ድ.አ ክላውድ9 ላይ አንድ አዲስ የስራ አካባብቢን መፍጠር።\label{fig:cloud9_new_workspace}}
\end{figure}

\begin{figure}
\begin{center}
\image{images/figures/ubuntu_server.png}
\end{center}
\caption{የ \textbf{Ubuntu Server} `ን መምረጥ።\label{fig:ubuntu_server}}
\end{figure}

\begin{figure}
\begin{center}
\image{images/figures/cloud9_ide_aws.png}
\end{center}
\caption{ነባሪው የደመና ቅ.ል.አ። \label{fig:cloud9_ide_aws}}
\end{figure}

በሩቢ ኮድን ለመጻፍ፣ በአንድ የጽሑፍ አርታኢ ውስጥ፣ በያንዳንዱ ኮድ መሃል፣ የሁለት \emph{ክፍተቦታ (Space)} መተው፣ ከሩቢ ልማዶች ውስጥ አንዱ ስለሆነ፣ እኔም የጽሑፍ አርታኢው፣ ነባሪውን የአራት ክፍተቦታ ከመጠቀም ይልቅ፣ የሁለት ክፍተቦታን ይጠቀም ዘንድ፣ የጽሑፍ አርታኢውን እንድትቀይሩት እመክራለሁ፡፡ በምስል~\ref{fig:cloud9_two_spaces} ላይ እንደተመለከተው፣ ከገጹ ራስጌ፣ በቀኝ በኩል የተመለከተውን ጥርሳማ ክብ አዶ ላይ ጠቅ በማድረግ፣ ከዛም በ ``Soft Tabs'' ቅንብር ላይ የተመለከተውን፣ የመቀነስ ምልክት ``2'' ቁጥር ላይ እስኪደርስ ድረስ ጠቅ በማድረግ፣ የተፈለገውን ውጤት ማግኘት ትችላላችሁ፡፡ (ይህ ቅንጅት በራስሰር ተግባራዊ ስለሚሆን ``Save'' የሚለውን አዝራር ጠቅ ማድረጉ አያስፈልግም፡፡)

\begin{figure}
\begin{center}
\image{images/figures/cloud9_two_spaces_aws.png}
\end{center}
\caption{ክላውድ9`ን ለግምስምስ የሁለት ክፍተቦታን እንዲጠቀም ማቀናበር።\label{fig:cloud9_two_spaces}}
\end{figure}

\begin{codelisting}
\label{code:upgrade_git}
\codecaption{ጊትን ማዘመን (አስፈላጊ ከሆነ)።}
%= lang:console
\begin{code}
$ git --version
# የስሪት ቁጥሩ ከ 2.28.0 ያልበለጠ ከሆነ የሚከተለውን ትእዛዝ አስኪዱ:-
$ source <(curl -sL https://cdn.learnenough.com/upgrade_git)
\end{code}
\end{codelisting}

በመጨረሻም፣ ይህ ስልጠና የሩቢ ስሪት ቁጥር 2.7.3 `ን እንደ አንድ መስፈርት አድርጎ የሚጠቀም ስለሆነ፣ ይህንኑ መስፈርት እንደሚከተለው አድርጋችሁ በደመና ቅ.ማ.አው ላይ መጫን ትችላላችሁ:-

%= lang:console
\begin{code}
$ rvm install 2.7.3
\end{code}

\noindent (ይህ በደመና ቅ.ማ.አው ላይ ቀድሞ-ተጪኖ የሚመጣውን \href{https://rvm.io}{የሩቢ ስሪት አስተዳዳሪን} ይጠቀማል።) አንዴ ትእዛዙ ከጨረሰ በኋላ፣ የሩቢ ስሪቱን እንደሚከተለው አድርጋችሁ ማረጋገጥ ትችላላችሁ:-

%= lang:console
\begin{code}
$ ruby -v
ruby 2.7.3p183 (2021-04-05 revision 6847ee089d) [x86_64-linux]
\end{code}

\noindent (እንቅጩ የስሪት ቁጥር ከላይ ከምታዩት የስሪት ቁጥር ሊለይ ይችላል)

በዚህ ጊዜ ጨርሳችኋል! ክላውድ9`ን ለመጠቀም የበይነመረብ መዳረሻ አስፈላጊ መሆኑን ልትገነዘቡ ይገባል።


% \section{Virtual machine} % (fold)
% \label{sec:virtual_machine}

% A second option for setting up a development environment is a \emph{virtual machine}, or \emph{VM}, which is a fully functional computer system that runs inside the host system. In the case of the Learn Enough VM that we recommend, you can run a full Linux system right inside of macOS or Windows (or even Linux! \href{https://en.wikipedia.org/wiki/Turtles_all_the_way_down}{It's turtles all the way down} (Figure~\ref{fig:turtles}).\footnote{Image retrieved from https://upload.wikimedia.org/wikipedia/commons/4/47/River\_terrapin.jpg on 2017-01-24. Image is in the public domain.}).

% \begin{figure}
% \begin{center}
% \includegraphics[width=5.5in]{images/figures/turtles.jpg}
% \end{center}
% \caption{It's turtles all the way down.\label{fig:turtles}}
% \end{figure}

% The virtual option we recommend is one we developed as part of \lecl, which involves installing a Linux virtual machine on your native system. The steps appear as follows:
% \begin{enumerate}
% \item Install the right version of \href{https://www.virtualbox.org/}{VirtualBox} for your system (free).
% \item Download the \href{https://softcover-static.s3.amazonaws.com/LearnEnough-v.1.4.ova}{Learn Enough Virtual Machine} (large file).
% \item Once the download is complete, double-click the resulting ``OVA'' file and follow the installation instructions.
% \item Double-click the VM itself and log in using the default user's password, which is ``\texttt{foobar!}''.
% \end{enumerate}
% (Getting all these steps to work is a good exercise in technical sophistication (Box~\ref{aside:technical_sophistication}).)

% The result of installing the VM is a Linux desktop environment (Figure~\ref{fig:virtual_machine}) that comes equipped with all the elements mentioned in Section~\ref{sec:dev_environment_options}, including a command-line terminal, the Atom text editor, Git, and Ruby. The interface might not be as familiar, fast, or as polished as your native OS, but the resulting development environment is industrial-strength and relatively easy to set up.

% \begin{figure}
% \begin{center}
% \image{images/figures/virtual_machine.png}
% \end{center}
% \caption{A Linux virtual machine running inside a host OS.\label{fig:virtual_machine}}
% \end{figure}

% section virtual_machine (end)


\section{የቤተኛ ስ.ክን ማዋቀር} % (fold)
\label{sec:native_os_setup}

በክፍል~\ref{sec:dev_environment_options} ላይ እንደተጠቀሰው፣ ቤተኛ ስርዓተ ክወናችሁን ለአንድ የማበልጸጊያ አካባቢ ማዋቀሩ ፈታኝ ሊሆን ይችላል፣ ነገር ግን ይሄ ውቅረት አንድ የተወሰነ የቴክኒካዊ ዘመናዊነት ደረጃ ላይ ከደረሱ በኋላ መውሰድ የሚገባው አንድ እርምጃ ነው፡፡ ካሉት አማራጮች ውስጥ የደመና ቅ.ማ.አ አማራጩ ይህንን ስራ ለመጀመር በጣም ጥሩው መንገድ ነው፣ ነገር ግን በሬውን በቀንዱ በመያዝ ቤተኛ ክወናውን እንደምትፈልጉት መግራት ይኖርባችኋል (ምስል~\ref{fig:grab_bull_by_horns})\footnote{ምስሉ በ 2017-01-24 ከ https://www.flickr.com/photos/mikey\_loves\_bcn/4354275361 የተወሰደ ነው፡፡ \href{https://www.flickr.com/photos/mikey_loves_bcn/}{በሚኪ ቪ} የ 2010 የቅጂ መብት © እና በ \ccbync\ ፈቃድ መሰረት ስእሉ ጥቅም ላይ ውሏል።}። (እዚህ ላይ በሬውን በቀንዱ ማያዝ አንድ የእንግሊዝኛ ፈሊጣዊ አነጋገር ሲሆን ትርጉሙ ደግሞ ``አስቸጋሪ ሁኔታን በቀጥታ ወይም ራስን በመተማመን መቋቋም ወይም መግጠም'' ማለት ነው።)

\begin{figure}
\begin{center}
\image{images/figures/grab_bull_by_horns.jpg}
\end{center}
\caption{አንዳንድ ጊዜ በሬውን በቀንዱ መያዝ ይኖርባችኋል።\label{fig:grab_bull_by_horns}}
\end{figure}

ክፍል~\ref{sec:macos} ማክስ.ክን (macOS) ወደ አንድ ሙሉ በሙሉ የታጠቀ የማበልጸጊያ አካባቢ መለወጥን የሚሸፍን ሲሆን፣ ክፍል~\ref{sec:linux} ደግሞ ለሊንክስ እንደዛው ያደርጋል፡፡ የማይክሮሶፍት ዊንዶውስ አማራጪን በክፍል~\ref{sec:windows} ላይ እንሸፍናለን፣ (በአጪሩ በክፍል~\ref{sec:dev_environment_options} ላይ እንደተጠቀሰው) ይህ ክፍል ግን በአሁኑ ወቅት በክፍል~\ref{sec:cloud_ide} ውስጥ የተጠቀሰውን የደመና ቅ.ማ.አ አማራጩን ይመለከታል፡፡

\subsection{ማክስ.ክ (macOS)} % (fold)
\label{sec:macos}

ቤተኛው የማኪንቶሽ ስርዓተ ክወና በመጀመሪያ ማክስ ማክ ኦኤስ~ኤክስ (Mac OS~X) ተብሎ ይጠራ ነበር፣ አሁን ግን ባጪሩ ማክኦኤክስ (macOS) ተብሎ ይጠራል፤ በአንድ ጠንካራ የዩኒክስ መሰረት ላይ ቢገነባም አንድ የተስተካከለ ስእላዊ በይነገጽ (ስ.በ GUI) አለው፡፡ በዚህ ምክንያትም፣ ማክኦኤክስ (macOS) ለአንድ አበልጻጊ እንደ የማበልጸጊያ አካባቢ ሁኖ ለማገልገል እጅግ ተስማሚ ሁኗል፡፡

በዚህ ክፍል ውስጥ ያሉት ሂደዶች አንድ አነስተኛ ስርዓትን ብቻ የሚመሰርቱ አይደሉም፤ በርግጥ ነው ስርዓቱን እናንተ ከዚህ ባነሰ መልኩ በመመስረት ባጪሩ ልትገላገሉ ትችላላችሁ፣ ነገር ግን ሶስቱም ደራሲዎቻችሁ ማክኦኤክስን ስለምንጠቀም እናንተን አንድ በደንብ ባልተሟላ መዋቀር መሸወዱ አስፈላጊ እንዳልሆነ ይሰማናል።

\subsubsection{መናኸሪያ እና አርታኢ} % (fold)
\label{sec:terminal_and_editor}

ምንም እንኳን ማክኦኤክስ ከአንድ ቤተኛ የመናኸሪያ ፕሮግራም ጋር አብሮ የሚመጣ ቢሆንም፣ እኛ ግን \href{https://www.iterm2.com/downloads.html}{አይተርምን} እንድትጪኑ እንመክራለን፣ ይህም ለገንቢዎች እና ለሌሎች የቴክኒክ ተጠቃሚዎች ከነባሪው መናኸሪያ የተሻለ እንዲሆን የሚያደርጉ \href{https://www.iterm2.com/features.html}{የተለያዩ መሻሻሎችን} ያካተተ ነው፡፡

እንዲሁም አንድ የአበልጻጊዎች የጽሑፍ አርታኢን እንድትጪኑ እንመክራለን። ብዙ ምርጥ የሆኑ ምርጫዎች ቢኖሩም \href{https://atom.io/}{የአተም አርታኢ} ግን ለመጀመሪያ ጥሩ ምርጫ ነው፤ ማለት ቀድሞውን የምትወዱት አንድ የጽሑፍ አርታኢ ከለላችሁ ማለት ነው (አተም \lete ላይ በደንብ ተሸፍኗል)።

ይህ በንዲህ እያለ፣ ለበቂ ተማሩ እና ለሬይልስ ማሰልጠኛ ትምህርት፣ ነባሪውን ማለት ዚ (Z) የተባለውን ቀፎ ከመጠቀም ይልቅ ቦርን-አጌይን የተባለውን ቀፎ እንድትጠቀሙ እንመክራለን (ምንም እንኳን ብዙውን ጊዜ በሁለቱ መካከል ምንም ለውጥ ባይኖረውም)። ቀፏችሁን ወደ ባሽ ለመቀየር፣ በትእዛዝ መስመሩ ላይ \kode{chsh -s /bin/bash} `ን አስኪዱ እና ከዚያ መሕለፈቃላችሁን ካስገባችሁ በኋላ የመናኸሪያ ፕሮግራማችሁን እንደገና አስጀምሩ። ይህንን በማድረግ የምታገኟቸውን የማስጠንቀቂያ መልዕክቶች ችላ ብላችሁ ብታልፏቸው የሚያመጡት ምንም ነገር የለም። ለበለጠ መረጃ በበቂ ይማሩ ብሎግ ልጥፍ ላይ ``\href{https://news.learnenough.com/macos-bash-zshell}{በበቂ ይማሩ የስልጠና ትምህርት ላይ የዚ ቀፎን በማክ ላይ መጠቀም}'' የሚለውን ተመልከቱ፡፡

% subsubsection terminal_and_editor (end)


\subsubsection{የኤክስኮድ የትእዛዝ መስመር መሳሪያዎች}
\label{sec:shiny_xcode}

ምንም እንኳን ማክኦኤክስ በዩኒክስ ላይ የተመሰረተ ቢሆንም፣\footnote{\href{https://en.wikipedia.org/wiki/NeXT}{ኔክስት (NeXT)} ስቲቭ ጆብስ ከአፕል ከተባረረ በኋላ እ.ኤ.አ በ 1985 ባቋቋመው ኩባንያ የተገነባ ስርዓት ነው። ኔክስት ኩባንያን አፕል እ.ኤ.አ በ 1997 ገዝቶ ከጠቀለለው ብኋላ፣ የኔክስት ስርዓተ ክወና (OS) የማክ ስርዓተ ክወና (OS) (በኋላ ላይ ደግሞ የማክኦኤክስ ስርዓተ ክወና) መሰረት ሆነ፤ ይህም ስቲቭን የአፕል ዋና ስራ አስኪያጅ ሁኖ በድል አድራጊነት እንዲመለስ ምክንያት ሆኗል።} ለአንድ የማበልጸጊያ አከባቢ አስፈላጊ ከሆኑት ሁሉ ሶፍትዌሮች ጋር ግን ተጪኖ አይሰደድም፡፡ ይህንን ለሟሟላት፣ የማክኦኤክስ ተጠቃሚዎች በአፕል የተፈጠሩ የማበልጸጊያ መሳሪያዎች እና የኮድ ቤተኮድ ስብስብ የሆነውን \emph{ኤክስኮድን (Xcode)} መጫን ይኖርባቸዋል።

ቀደም ሲል ኤክስኮድ፣ አንድ ከ 4 ጌጋባይት በላይ የሆነ የምንጪ ፋይሎችን መጫኛን ማውረድ ይፈልግ ነበረ፣ አፕል እግዚአብሄር ይስጠው በቅርብ ጊዜ ዝርዝር~\ref{code:xcode-install} ላይ እንደሚታየው በአንድ ቀላል የትእዛዝ መስመር ኤክስኮድን መጫኑን እጅግ ፈጣን እና ቀላል አድርጎታል፡፡

\begin{codelisting}
\label{code:xcode-install}
\codecaption{የኤክስኮድ የትእዛዝ መስመር መሳሪያወችን መጫን።}
%= lang:console
\begin{code}
$ xcode-select --install
\end{code}
\end{codelisting}


\subsubsection{ሆምብሬው}
\label{sec:homebrew}

ቀጣዩ ሂደት በምርጫ ሊደረግ የሚችል ነው፣ ነገር ግን በእኛ አመለካከት አንድ የባለ ሙያ ደረጃ ላለው እውነተኛ የማክኦኤክስ የማበልጸጊያ አካባቢ አስፈላጊ ነው ብለን እናምናለን፣ እሱም ግሩሙን \emph{የሆምብሬው (Homebrew)} የጥቅል አስተዳዳሪን መጫን ነው።

አንድ የጥቅል አስተዳዳሪን ልክ በነጻ የክፍተ-ምንጪ ሶፍትዌር እንደተሞላ እና በትእዛዝ መስመር እንደሚካሄድ የአፕልኬሽን ማከማቻ አድርጋችሁ ማሰብ ትችላላችሁ፡፡ በአሁኑ ጊዜ አብዛኛዎቹ የሊኒክስ ስርጪቶች ከአንድ ቤተኛ የጥቅል አስተዳዳሪ (ክፍል~\ref{sec:linux}) ጋር አብረው ይመጣሉ፣ ይሁን እንጅ ማክኦኤክስ በነባሪነት ምንም የለውም፡፡ ሆምብሬው በክፍተ-ምንጪ ማህበረሰብ ውስጥ ከሚገኙ ብዙ የጥቅል አስተዳዳሪዎች መካከል አንዱ ነው፣ ከጊዜ በኋላ ግን በብዙ የማክኦኤክስ ገንቢዎች መካከል በጣም ተወዳጅነትን ያተረፈ አማራጪ ሆኗል፡፡

በክፍል~\ref{sec:install_ruby} ላይ እንደምናየው፣ ሩቢን ለመጫን አንድ \emph{አርቢኤንቭ} የተባለ ፕሮግራምን እንጠቀማለን፣ እሱም በክፍል~\ref{sec:rbenv} ውስጥ በሆምብሬው በኩል ይጫናል፡፡ ይህ በንዲህ እያለ ሆምብራው ራሱ ሩቢን ይፈልጋል፣ ስለሆነም አንድ \href{https://en.wikipedia.org/wiki/Circular_dependency}{የጥገኝነት ክብ} ውስጥ የገባን ይመስላል። ደስ የሚለው ነገር ደግሞ፣ ማክኦኤክስ ጪነቱን \href{https://en.wikipedia.org/wiki/Bootstrapping}{በራሱ-የሚጀምር ሂደት} አድርገን ልንጠቀምበት ከምንችልበት አንድ \emph{የሩቢ ስርዓት} ጋር ተጪኖ መሰደዱ ነው፡፡ ሆምብሬውን ለመጫን ይህንን ነባሪ ሩቢ እንጠቀማለን፣ ከዚያ ከላይ እንደተጠቀሰው አርቢኤንቭን እና ተጨማሪ የሩቢ ስሪቶችን እንጪናለን።

የሆምብሬው መጫኛ ፕሮግራሙ አንድ \href{https://www.learnenough.com/text-editor-tutorial/advanced_text_editing#sec-writing_an_executable_script}{የባሽ ፕሮግራም}ሲሆን፣ \lecl ላይ \href{https://www.learnenough.com/command-line-tutorial#sec-downloading_a_file}{እንደተሸፈነው} \kode{curl} የተባለውን ፕሮግራም በመጠቀም ሊደረስበት ይችላል፡፡ በ \kode{/bin} ማውጫ ውስጥ የሚገኘውን፣ የ \kode{bash} ስርዓት ተፈጻሚን በመጠቀም፣ የሆምብሬው መጫኛ ፕሮግራምን ማስፈጸም እንችላለን። ሙሉው ትእዛዝ በዝርዝር~\ref{code:homebrew-install} ላይ እንደሚታየው ይሆናል።

\begin{codelisting}
\label{code:homebrew-install}
\codecaption{የሆምብሬው ጥቅል አስተዳዳሪን መጫን።}
%= lang:console
\begin{code}
$ /bin/bash -c "$(curl -fsSL https://www.learnenough.com/homebrew.sh)"
\end{code}
\end{codelisting}

\noindent ዝርዝር~\ref{code:homebrew-install} ወቅታዊውን የሆምብሬው መጫኛ ፕሮግራምን የሚያመላክት አንድ የ learnenough.com ማስተላለፊያ ዓ.አ.ሃ.አን (URL) እንደሚጠቀም ልታስተውሉ ይገባል። በዚህ መንገድ፣ ሆምብሬው ፕሮግራሙን ለማስተናገድ የሚያገለግለውን ዓ.አ.ሃ.አን ቢቀይረው እኛም የማስተላለፊያ አድራሻውን በቀላሉ ማዘመን እንችላለን፣ ስለሆነም ይህ የስልጠና ትምህርት ሁልጊዜ ወቅታዊነቱን እንደጠበቀ መስራቱን ይቀጥላል ማለት ነው። (ከፈለጋችሁ ደግሞ፣ \href{http://brew.sh/}{ከሆምብሬው የመነሻ ገጽ} ላይ የሚገኘውን ሙሉውን ትእዛዝ ቀድታችሁ በትእዛዝ መስመራችሁ ላይ መገልበጥ ትችላላችሁ።)

ሆምብሬው ጥቅሎችን ለመጫን፣ ለማዘመን እና ለማስወገድ የሚያገለግል አንድ \kode{brew} የተባለ የትእዛዝ መስመርን ይጪናል። ሆምብሬው መጫኑን ከጨረሰ በኋላ፣ የሰፈር ፋይሎችን ለማስተዳደር በሆምብሬው የሚፈለጉ ሁሉም ማውጫወች እና ፈቃዶች በትክክል መዘጋጀታቸውን የሚያረጋግጠውን የ \kode{brew doctor} ትእዛዝን ማስኬዱ አንድ ጥሩ ሀሳብ ነው:-

%= lang:console
\begin{code}
$ brew doctor
\end{code}

\noindent በዚህ ጊዜ የሆነ ችግር ካጋጠማችሁ \href{https://github.com/Homebrew/homebrew/wiki/troubleshooting}{በዊኪ የሆምብሬው መላ መፈለጊያ ላይ} መጣቀስ ይኖርባችኋል፣ በእውነቱ ከሆነ ግን የስርዓት አቃፊዎችን በዘፈቀደ ካለወጣችሁ እና የፈቃድ ችግር ካላጋጠማችሁ በስተቀር ምንም አይነት ችግር ሊያጋጥማችሁ አይገባም።

\subsubsection{የሩቢ አካባቢ (rbenv)}
\label{sec:rbenv}

ቀደም ሲል ማክኦኤክስ ሩቢ ጋር አብሮ ተጪኖ እንደሚመጣ ተመልክተናል፣ በስሪቱ እንቅጪ ቁጥር ላይ ግን ምንም አይነት የመቆጣጠር ሃይሉ የለንም፣ ማክኦኤክስም በቤተኛነት በርካታ የሩቢ ተጓዳኝ ስሪቶችን እንድንጠቀም አያስችለንም። የማበልጸጊያ አካባቢያችንን ልባችን እንደፈቀደው እንድናደርግ ያስችለን ዘንድ፣ \emph{አርቢኤንቭን (rbenv)} እንጪናለን። ይህም የተለያዩ የሩቢ ስሪቶችን የሚያስተዳድር እና የሩቢ የሶፍትዌር ጥቅሎችን (ማለት \emph{እንቁዎች (gems)} ተብለው የሚጠሩትን) ሩቢ እንዲያገኛቸው በትክክለኛው ቦታ ላይ እንደተቀመጡ የሚያረጋግጥ አንድ መገልገያ\footnote{(መገልገያ (Utility) ዘዴ ማለት፣ መደበኛ የፕሮግራም ስራዎችን ለማከናወን ጠቃሚ የሆነ፣ በተደጋጋሚ የሚደረግ የፕሮግራም ስራን ለመፈጸም የሚያስችል፣ በድጋሜ ሊጠቀሙበት የሚችሉ ዘዴ/ፕሮግራም ማለት ነው)} ነው።

አርቢኤንቭን ከ \emph{ruby-build} ትእዛዝ ጋር አብሮ መጠቀሙ፣ ለተለያዩ የፕሮጀክት ማከማቻዎች አንድ የተለየ የሩቢ ስሪትን እንድንለይ ያስችለናል፤ ይህም በሶፍትዌር ብልጸጋ ላይ አንድ የተለመደ ተግባር ነው፡፡\footnote{ለምሳሌ ያየን እንደሆን \href{http://docs.python-guide.org/en/latest/dev/virtualenvs/}{\emph{ቨርቹዋልኤንቭ (virtualenv)}} የተባለው መገልገያ ፓይቶን ለተባለው የፕሮግራም ቋንቋ እንደዚሁ ለተለያዩ ፕሮጄክቶች የሚያገለግል ተመሳሳይ ስራን ያከናውናል።} ለምሳሌ፣ አንድ የቆየ የፕሮግራም ስሪትን በትክክል ለማስኬድ አንድ የቆየ የሩቢ ስሪትን ሊፈልግ ይችላል። ይህ ማለት አርቢኤንቭን በመጠቀም፣ ሌሎች ፕሮጀክቶቻችን አንድ በጣም ወቅታዊ በሆነ የሩቢ ስሪት እያስኬድን እንዲህ አይነቱን የቆየ አንድ ፕሮግራም ደግሞ በአንድ የቀድሞ የሩቢ ስሪት ማስኬድ እንችላለን ማለት ነው።

በሆምብሬው አርቢኤንቭን መጫኑ እጅግ ቀላል ነው። በዝርዝር~\ref{code:rbenv} ላይ እንደሚታየው \kode{brew install rbenv} `ን በመጠቀም ሁለቱንም ማለት አርቢኤንቭን እና የሩቢ-ገንቢን (ruby-build) በአንድ ጊዜ መጫን እንችላለን፡፡

\begin{codelisting}
\label{code:rbenv}
\codecaption{አርቢኤንቭን እና የሩቢ-ገንቢን መጫን።}
%= lang:console
\begin{code}
$ brew install rbenv   # እንዲሁም የሩቢ-ገንቢን (ruby-build) በራስሰር ይጪናል።
\end{code}
\end{codelisting}

ዝርዝር~\ref{code:rbenv} መጫኑን ከጨረሰ በኋላ፣ በዝርዝር~\ref{code:rbenv_init} ላይ እንደሚታየው \kode{rbenv init} `ን በመጠቀም አርቢኤንቭን በስራ ላይ ማዋል ይፈለግብናል፡፡

\emph{ማሳሰቢያ}:- የዚኤስኤች (Zsh) ቀፎን የምትጠቀሙ ከሆነ፣ \kode{.zshrc} `ን በ \kode{.bash\_profile} መተካት ይኖርባችኋል። የበለጠ መረጃ ለማግኛት ``\href{https://news.learnenough.com/macos-bash-zshell}{በበቂ ይማሩ የስልጠና ትምህርት ላይ የዚ ቀፎን በማክ ላይ መጠቀም}'' የሚለውን ተመልከቱ፡፡

\begin{codelisting}
\label{code:rbenv_init}
\codecaption{አርቢኤንቭን መጫን።}
%= lang:console, options: "hl_lines": [1]
\begin{code}
$ rbenv init
# Load rbenv automatically by appending
# the following to ~/.bash_profile (or ~/.zshrc if using Zsh):

eval "$(rbenv init -)"
\end{code}
\end{codelisting}

\noindent ዝርዝር~\ref{code:rbenv_init} ``እንደዚህ አይነት ፋይል ወይም ማውጫ የለም (No such file or directory.)'' የሚል አንድ የስህተት መልእክት ከሰጣችሁ፣ Ctrl-D `ን በመጠቀም ከቀፎ ፕሮግራሙ ውጡ እና እንደገና አስጀምራችሁ ከዚያ ትእዛዙን እንደገና ሙክሩት። (ይህ ዓይነቱ እንደገና የማስጀመር እና እንደገና የመሞከሩ ዘዴ፣ የተለመደ ቴክኒካዊ ብልሃት ነው (ሳጥን~\ref{aside:technical_sophistication})።)

በዝርዝር~\ref{code:rbenv_init} ላይ እንደተመለከተው፣ \kode{rbenv init} `ን ማስኬዱ፣ ሁልጊዜ አርቢኤንቭን እራሳችን ከማስጀመር ይልቅ ይህንን ትእዛዝ:-

%= lang:bash
\begin{code}
eval "$(rbenv init -)"
\end{code}

\noindent በ \kode{.bash\_profile} ማህደር ውስጥ በማስገባት ድግግሞሽን እንዴት ማስወገድ እንደምንችል አንድ አስተያየትን ያበረክታል (ይህም \lete ላይ መደንብ \href{https://www.learnenough.com/text-editor-tutorial#sec-saving_and_quitting_files}{ተሸፍኗል}።)

ከፈለጋችሁ በናንተ የ \kode{.bash\_profile} ማህደር ውስጥ \kode{eval} የሚለውን ትእዛዝ ለማከል አንድ የጽሑፍ አርታኢን መጠቀም ትችላላችሁ፣ \lecl ላይ እንደተሸፈነው ቀላሉ መንገድ ግን \kode{echo} `ን እና የቅጥያ~\kode{>{}>} ስሌትን እንዲህ አድርጎ መጠቀም ነው:-

%= lang:console
\begin{code}
$ echo 'eval "$(rbenv init -)"' >> ~/.bash_profile
# ዚኤስኤችን (Zsh) የምትጠቀሙ ከሆነ ~/.zshrc `ን ተጠቀሙ
\end{code}

\noindent እዚህ ላይ፣ በአሁኑ ወቅት በየትኛውም ማውጫ ውስጥ ብንሆንም እንኳን ትእዛዙ እንዲሰራ፣ የቤት ማውጫውን \kode{\textasciitilde} በመንገዱ ውስጥ እንዳካተትነው ልታስተውሉ ይገባል፡፡


በመጨረሻም፣ (\lete ላይ \href{https://www.learnenough.com/text-editor-tutorial#code-source_command}{እንደተጠቀሰው}) አዲሱን የመገለጫ ማህደር ለማግበር \kode{source} የተባለውን ትእዛዝ መጠቀም ይኖርብናል:-

%= lang:console
\begin{code}
$ source ~/.bash_profile  # ዚኤስኤችን (Zsh) የምትጠቀሙ ከሆነ ~/.zshrc `ን ተጠቀሙ
\end{code}


\subsubsection{አዲስ የሩቢ ስሪት}
\label{sec:install_ruby}

አሁን አርቢኤንቭ ስለተዘጋጀ፣ እስኪ አንድ ስርዓት ውስጥ ያልገባ የሩቢ ስሪትን እንዲያስተዳር እንስጠው። የመጫን ሂደቱ ሙሉ በሙሉ የሚስተናገደው በአርቢኤንቭ ነው፣ ስለሆነም ማድረግ ያለባችሁ ነገር ቢኖር  ትክክለኛውን የሩቢ የስሪት ቁጥር ከአርቢኤንቭ ትእዛዝ ጋር ጎንለጎን በመስጠት በስርዓታችሁ ላይ የትኛውን ስሪት እንደምትፈልጉ መንገር ብቻ ነው፡፡

በዚህ የስልጠና ትምህርት ውስጥ ሩቢ 2.7.3 `ን እንጠቀማለን፣ ይህ ጽሑፍ እስከተጻፈበት ጊዜ ድረስ የተለያዩ የሩቢ አፕልኬሽኖች ላይ በደንብ እንደሚሰራም ይታወቃል፣ ነገር ግን ከፈለጋችሁ በሩቢ ድርጣቢያ ላይ እንደተዘረዘረው \href{https://www.ruby-lang.org/en/downloads/}{ወቅታዊውን ወይም የቀድሞውን} የሩቢ ስሪት መጠቀምም ትችላላችሁ፡፡

\kode{rbenv} `ን በመጠቀም የተፈለገውን የሩቢ ስሪት ለመጫን፣ በዝርዝር~\ref{code:ruby-install} ላይ የሚታየውን ትእዛዝ ብቻ ፈጽሙ፡፡ ስርዓታችሁ የተሰጠው የሩቢ ስሪት አይገኝም ብሎ ቅሬታ ካሰማ፣ ወቅታዊውን ስሪት ለመድረስ ስርዓታችሁን ማዘመን ይኖርባችኋል (ሳጥን~\ref{aside:updating_upgrading})።

\begin{codelisting}
\label{code:ruby-install}
\codecaption{አንድ አዲስ የሩቢ ቅጂን መጫን።}
%= lang:console
\begin{code}
$ rbenv install 2.7.3
\end{code}
\end{codelisting}

\noindent በዝርዝር~\ref{code:ruby-install} ውስጥ ያለውን ትእዛዝ ካስኬዳችሁ በኋላ፣ አርቢኤንቭ የማውረድ ሂደቱን ሲጀምር እና ለዚያ የተወሰነ የሩቢ ስሪት የሚያስፈልጉትን ጥገኛዎች (ጥገኛ ፕሮግራሞች) ሲጪን ማየት አለባችሁ፡፡ (ይህም በመተላለፊያ ይዘቱ እና በማዕከላዊ የሂደት ክፍሉ (ማ.ሂ.ክ (CPU)) ወሰን ላይ በመመርኮዝ ትንሽ ጊዜ ሊወስድ ይችላል)።



\emph{ማሳሰቢያ}:- ይህን የመሰለ አንድ የስህተት መልእክት ልታገኙ ትችሉ ይሆናል:-

%= lang:console
\begin{code}
ruby-build: definition not found: 2.7.3

See all available versions with `rbenv install --list'.

If the version you need is missing, try upgrading ruby-build:

  brew update && brew upgrade ruby-build
\end{code}

\noindent ይህ ከተከሰተ እጅግ ወቅታዊውን የ \kode{ruby-build} ዝርዝርን በአርቢኤንቭ \kode{plugins} ማውጫ ውስጥ መሰል\footnote{\kode{git clone} አንድ ነባር ማከማቻ ላይ ኢላማ በማድረግ ኢላማ የተደረገበትን ማከማቻ ለመቅዳት የሚያገለግል አንድ የጊት የትእዛዝ መስመር መገልገያ ነው፡፡ ስለዚህ በጊት ውስጥ መሰል (Clone) ማድረግ ማለት ከአንድ ማከማቻ ላይ መቅዳት ማለት ነው።} (Clone) ማድረግ ይኖርባችኋል:-

%= lang:console
\begin{code}
$ mkdir ~/.rbenv/plugins
$ git clone https://github.com/rbenv/ruby-build.git ~/.rbenv/plugins/ruby-build
\end{code}

\noindent \kode{ruby-build} ቀድሞውኑ እዛው ላይ ካለ፣ ወቅታዊ ለውጦቹን በቀላሉ ለማውረድ የጊት \kode{pull} ትእዛዝን እንደሚከተለው አድርጋችሁ መጠቀም ትችላላችሁ:-

%= lang:console
\begin{code}
$ cd ~/.rbenv/plugins && git pull && cd -
\end{code}

\begin{aside}
\label{aside:updating_upgrading}
\heading{ማዘመን እና ማሻሻል}

ምንም እንኳን አሁን ሁሉንም ማለት ተገቢ የሆነ ግንኙነት ያላቸውን የሶፍትዌሮች ወቅታዊ ስሪቶችን ብጪንም፣ ከጥቂት ጊዜ በኋላ ግን የሰፈራችሁ ስርዓት ከአዳዲስ የሆምብሬው ስሪቶች እና ከማንኛውም የተጫኑ ጥቅሎች ጋር አይመሳሰሉም፡፡ ስርዓቱን ለማዘመን፣ አንዳንድ ጊዜ ሆምብሬውን እራሱን ማዘመን እና ከዚያ የተጫኑትን ጥቅሎችንም ማሻሻሉ አንድ ጥሩ ሀሳብ ነው:-

\begin{verbatim}
  $ brew update
  $ brew upgrade
\end{verbatim}

\end{aside}

ሩቢን መጫኑ ከተጠናቀቀ በኋላ፣ ግልጽ ባልሆነ ስም የተሰየመውን የ \kode{rehash} ትእዛዝን በመጠቀም አንድ አዲስ የሩቢ ስሪት እንዳለ ለስርዓቱ መንገር ይኖርብናል:-

%= lang:console
\begin{code}
$ rbenv rehash
\end{code}

ለዚህ መመሪያ፣ ፕሮጀክታችሁን በምትጀምሩበት ጊዜ የሩቢ ስሪትን ለይታችሁ ለመጥቀስ እንዳትጨነቁ፣ የዝርዝር~\ref{code:ruby-install} የሩቢ ስሪቱን በተጨማሪ አንድ የዓለም አቀፍ ነባሪ ስሪት እንዲሆን እናደርጋዋለን (እዚህ ላይ አንድ ስሪትን ዓለም አቀፍ ማድረግ ማለት በዚህ ስርዓት ውስጥ የሚፈጠሩ አፕልኬሽኖች በሙሉ አንድ ዓለም አቀፍነት የተሰጠውን ስሪት ይጠቀማሉ ማለት ነው፤ በዚህ ወቅት ደግሞ የሩቢ የስሪት ቁጥር 2.7.3 `ን ይሆናል ማለት ነው)፡፡ ይህንን ለማድረግ መንገዱም የ \kode{global} ትእዛዝን መጠቀም ነው:-

%= lang:console
\begin{code}
$ rbenv global 2.7.3
\end{code}

\noindent በዚህ ጊዜ ሁሉም ቅንጅቶች በትክክል እንደተዘመኑ ለማረጋገጥ፣ የቀፎ ፕሮግራሙን እንደገና ማስጀመሩ አንድ ጥሩ ሀሳብ ሊሆን ይችላል፡፡

ለወደፊቱ ስራ፣ በአንድ ፕሮጀክት መሰረት ላይ የተወሰነ የሩቢ ስሪትን መጠቀም ትፈልጉ ይሆናል፣ ይህም በፕሮጀክቱ ስረ ማውጫ ላይ አንድ \kode{.ruby-version} የተባለ ፋይልን በመፍጠር እና ጥቅም ላይ እንዲወል የተፈለገውን የሩቢ ስሪትን በማካተት ሊከናወን ይችላል። (በስርዓታችሁ ላይ አሁን ከሌለ፣ በዝርዝር~\ref{code:ruby-install} ላይ እንደተመለከተው \kode{rbenv install <የስሪት ቁጥር>} `ን በመጠቀም መጫን ይኖርባችኋል።) የበለጠ መረጃ ለማግኘት \href{https://github.com/rbenv/rbenv}{የአርቢኤንቭ ሰነድን} ተመልከቱ።

በመጨረሻም፣ የሩቢ ሶፍትዌርን \emph{በእንቁዎች (gems)} ወይም እራሱን-በያዘ የሩቢ ኮድ ጥቅል በኩል በምትጪኑበት ጊዜ፣ ሶፍትዌሩን ከመጫን ይልቅ የሰፈር ሰነዱን ለመጫን ብዙ ጊዜ ሊወስድ ስለሚችል፣ ብዙውን ጊዜ እሱን ላለመጫን መዝለሉ የተመረጠ ነው፣ በተፈለገ ጊዜ ደግሞ ሰነዱን በመስመር ላይ መድረሱ እጅግ አመች ነው። ሰነድ መጫንን መከላከሉ፣ እንደየሁኔታው ሊከናወን ይችላል፣ ነገር ግን በዝርዝር~\ref{code:gemrc} ላይ እንደሚታየው፣ እምብዛም ጥቅም ላይ የማይውሉትን (እና ለመጫን ጊዜ የሚወስዱትን) የሩቢ የሰነድ ፋይሎች ለመዝለል በቤት ማውጫው ውስጥ አንድ \kode{.gemrc} ተብሎ የሚጠራ ፋይልን በመፍጠር እሱን አንድ ዓለም አቀፍ ነባሪ ማድረጉ የበለጠ አመች ይሆናል።

\begin{codelisting}
\label{code:gemrc}
\codecaption{የሩቢ የሰነዶች ጪነትን ለመዝለል \kode{.gemrc} ማህደርን ማዋቀር።}
%= lang:console
\begin{code}
$ echo "gem: --no-document" >> ~/.gemrc
\end{code}
\end{codelisting}


\noindent በዚህ ውቅር የሩቢ እንቁዎችን ለመጫን፣ ማንኛውም \kode{gem install <የእንቁ ስም>} `ን መጠቀሙ፣ በራስሰር ስልክክ ያለ፣ ቀልጣፋ እና ከሰነድ ነጻ እንዲሆን ያደርገዋል፡፡

\subsubsection{ጊት} % (fold)
\label{sec:git}

ወቅታዊው የጊት የስሪት ቁጥጥር ስርዓት ክፍል~\ref{sec:shiny_xcode} ላይ ከተጫኑት የኤክስኮድ የትእዛዝ መስመር መሳሪያዎች ጋር በራስሰር መምጣት ይኖርበታል፣ ይህንንም የ \kode{which} ትእዛዝን በመጠቀም ማረጋገጥ ትችላላችሁ:-

%= lang:console
\begin{code}
$ which git
\end{code}

\noindent የዚህ ውጤት ባዶ ከሆነ፣ ጊት አልተጫነም ማለት ነው፣ እናም ሆምብሬውን በመጠቀም እሱን መጫን ትችላላችሁ:-

%= lang:console
\begin{code}
$ brew install git
\end{code}

ያም ሆነ ይህ፣ የጊት የስሪት ቁጥር ቢያንስ ቢያንስ \kode{2.28.0} መሆኑን ማረጋገጥ ይኖርባችኋል:-

%= lang:console
\begin{code}
$ git --version
git version 2.31.1    # የስሪት ቁጥሩ ቢያንስ ቢያንስ 2.28.0 መሆን አለበት።
\end{code}

\noindent የስሪት ቁጥሩ ወቅታዊ ካልሆነ፣ መጀመሪያ \kode{brew update} `ን ከዚያ ደግሞ \kode{brew upgrade git} `ን ማስኬድ ይኖርባችኋል።

% subsubsection git (end)

% subsection macos (end)

\subsection{ሊኒክስ} % (fold)
\label{sec:linux}

በሊኒክስ እጅግ ቴክኒካዊ የዘር አመጣጥ ምክንያት፣ የሊኒክስ ስርዓቶች በተለምዶ በአፕልኬሽኖች መገንቢያ መሳሪያዎች በሚገባ የተሰነቁ ናቸው። በዚህ ምክንያት፣ አንድ ቤተኛ የሊኒክስ ስርዓተ ክወናን ለአንድ የማበልጸጊያ አካባቢ ለማዘጋጀት በሚገባ ቀላል ነው፡፡

እያንዳንዱ ዋና የሊኒክስ ስርጪት ከአንድ የመናኸሪያ ፕሮግራም፣ ከአንድ የጽሁፍ አርታኢ እና ከጊት ጋር ተጪኖ ይሰደዳል፡፡ ከነባሪዎቹ በተጨማሪ የምንመክራቸው ነገሮች ቢኖሩ ግን እነዚህ አራት ዋና ሂደቶች ብቻ ይሆናሉ:-
\begin{enumerate}
  \item ቀድሞውኑ የምትወዱት አንድ የጽሑፍ አርታኢ ከለላችሁ፣ \href{https://atom.io/}{አተምን አውርዱ እና ጫኑ}።
  \item የአርቢኤንቭ \href{https://github.com/rbenv/rbenv#installation}{የጪነት መመሪያዎችን} ከአርቢኤንቭ የድርጣቢያ ተከተሉ።
  \item በዝርዝር~\ref{code:ruby-install} እና በዝርዝር~\ref{code:gemrc} ላይ እንደታየው ሩቢን ጫኑ እና አዋቅሩ።
  \item የጊት የስሪት ቁጥር ቢያንስ ቢያንስ \kode{2.28.0} መሆኑን አረጋግጡ።
\end{enumerate}

ከላይ ካሉት ውስጥ የመጨረሻው ሂደት የሚከተለውን ትእዛዝ በማስኬድ ሊከናወን ይችላል:-

%= lang:console
\begin{code}
$ git --version
\end{code}

\noindent የስሪቱ ውጽአት ቢያንስ ቢያንስ \kode{2.28.0} ካልሆነ፣ ወደ  ``\href{https://git-scm.com/book/en/v2/Getting-Started-Installing-Git}{ጊትን ለመጫን መጀመር}'' አምሩ እና ወቅታዊውን የጊት ስሪት ለስርዓታችሁ ጫኑ።

በዚህ ጊዜ፣ ሁሉንም ነገር በትክክል አሟልታችሁ ወደሚቀጥለው ጉዞ ለማለፍ መብቃት አለባችሁ!

% subsection linux (end)

\subsection{ዊንዶውስ} % (fold)
\label{sec:windows}

በመጨረሻም፣ ለማይክሮሶፍት ዊንዶውስ አንድ የቤተኛ መመሪያ ይኖረናል፣ ወይም በትክክለኛው አነጋገር በዊንዶውስ ላይ ዩኒክስን የሚጠቀሙ መመሪያወች አሉን ማለት ይቻላል። በክፍል~\ref{sec:cloud_ide} ላይ እንደተብራራው፣ አንዱ አማራጪ አንድ የደመና ቅ.ማ.አን መጠቀም ነው፣ በተለይ በዊንዶውስ ላይ ሊኒክስን በቀጥታ መጫኑ ግን መልካም ውጤት እንዳለው አንዳንድ ዘገባወች ደርሰውናል፡፡

ትክክል ነው! እመኑም አትምኑም፣ አሁን ዊንዶውስ ከአንድ ከሚሰራ የሊኒክስ \href{https://en.wikipedia.org/wiki/Kernel_(operating_system)}{ከርኔል (kernel)} ጋር \href{https://devblogs.microsoft.com/commandline/announcing-wsl-2/}{ተጪኖ ይሰደዳል}፣ እና \href{https://docs.microsoft.com/en-us/windows/wsl/install-win10}{የራሱን የማይክሮሶፍትን መመሪያዎች} በመከተል ማንኛውንም የሊኒክስ ስርጪቶችን መጫን ትችላላችሁ፡፡ (በ 90 ዎቹ መጨረሻ እና በ 2000 ዎቹ መጀመሪያ ላይ የሊኒክስ የማይክሮሶፍት ጥላቻን እና የማይክሮሶፍት ስግብግብነትን/ቀማኝነትን ለምናስታውስ፣ ዊንዶውስ አንድ ቀን አንድ ቤተኛ ሊኒክስን ከመገደፍ ጋር ተጪኖ ይሰደዳል ብሎ ማሰቡ በእውነቱ ጪራሽ የማይታሰብ ነገር ነበረ፣ \href{https://docs.microsoft.com/en-us/windows/wsl/install-win10}{አሁን ግን ለዚህ በቃን} (\href{https://youtu.be/JmzuRXLzqKk}{\emph{ውሻ እና ድመት አብረው ይኖራሉ ይሏችኋል ይሄ ነው!}})፡፡)

ሁሉም ነገር እንዲሰራ ለማድረግ፣ ምክራችን \href{https://www.hanselman.com/}{በስካት ሀንሰልማን} የቀረበውን ``\href{https://www.hanselman.com/blog/RubyOnRailsOnWindowsIsNotJustPossibleItsFabulousUsingWSL-8AndVSCode.aspx}{ሩቢ ኦን ሬይልስን በዊንዶውስ መገንባት የሚቻል ብቻ ሳይሆን፣ ደብሊውኤስኤል2`ን (WSL2) እና ቪኤስ ኮድን (VS Code) በመጠቀም መገንባቱ እጅግ ድንቅ ነው}'' የሚለውን ትምህርታዊ መጣጥፍን እንድትከተሉ ነው። ምንም እንኳን ለሩቢ ኦን ሬይልስ የድር ማበልጸጊያ ጠቃሚ ቢሆንም፣ የሀንሰልማን የስልጠና ትምህርት ግን በአጠቃላይ ብዙ ዘርፎች ላይ የሚያገለግል ነው፣ እና የስልጠና ትምህርቱን መከታተሉ አንድ ድንቅ የዊንዶውስ ስርዓት ማበልጸጊያ አጠቃላይ ተሞክሮን ሊያስገኝ ይገባል፡፡

% subsection windows (end)

% section native_os_setup (end)

\section{ማጠቃለያ} % (fold)
\label{sec:conclusion}

እስከዚህ ድረስ ከደረሳችሁ (እና በተለይ ደግሞ በክፍል~\ref{sec:native_os_setup} ላይ ያለውን አንድ የቤተኛ ስርዓተ ክወና ማዋቀርን ካጠናቀቃችሁ) አሁን ``በማበልጸጊያ አካባቢ \emph{አደገኛ ለመሆን} በቂ ይማሩ'' `ን ተምራችኋል ማለት ነው። አሁን እንደ \lecss\/፣ \ler እና እንደ \rort ያሉ ፈታኝ የስልጠና ትምህርቶችን ለመፋለም እና በድል ለማጠናቀቅ ዝግጁዎች ናችሁ፤ መልካም ዕድል!

% section conclusion (end)

\bigskip

\noindent {\small \ledev ። የቅጂ መብቱ © እ.ኤ.አ በ 2017 በማይክል ሃርትል፣ በሊዶናሆ እና በኒክ ሜርዊን ተሰጥቷል።}
